\UseRawInputEncoding
%\begin{tabular}{|c|c|c|c|c|c|c|}
%\hline
%& \multicolumn{2}{c}{�rbita 1}&\multicolumn{2}{|c}{�rbita 2}&\multicolumn{2}{|c|}{�rbita 3}\\
%\hline
%& $x$ & $y$ & $x$ & $y$ & $x$ & $y$ \\
%\hline
%$z_{0}$ & 1.00 & 0.00 & 0.50 & 0.25 & 0.00 & 0.88\\
%$z_{1}$ & 0.50 & 0.50 & -0.31 & 0.75 & -1.27 & 0.50\\
%$z_{2}$ & -0.50 & 1.00 & -0.96 & 0.03 & 0.87 & -0.77\\
%$z_{3}$ & -1.25 & -0.50 & 0.43 & 0.44 & -0.34 & -0.85\\
%$z_{4}$ & 0.81 & 1.75 & -0.51 & 0.88 & -1.12 & 1.07\\
%$z_{5}$ & -2.90 & 3.34 & -1.01 & -0.39 & -0.41 & -1.90\\
%$z_{6}$ & -3.26 & -18.91 & 0.37 & 1.30 & -3.93 & 2.04\\
%$z_{7}$ & -347.46 & 123.68 & -2.04 & 1.46 & 10.79 & -15.52\\
%$z_{8}$ &  &  & 1.53 & -5.46 & -124.77 & -334.49\\
%$z_{7}$ &  &  & -28.01 & -16.27 &  & \\
%\hline
%\end{tabular}
%\caption{\textbf{Tres punts de fugida.} La iteraci� de tres punts inicials per a $z\rightarrow z^{2}+c$, $c=$-0.5+0.5$i$. Les tres �rbites escapen a l'infinit.}
\begin{tabular}{lrrr}
\toprule
$z$ & �rbita 1 & �rbita 2 & �rbita 3 \\
\midrule
$z_0$ & 1.0000+0.0000$i$ & 0.5000+0.2500$i$ & 0.0000+0.8800$i$ \\
$z_1$ & 0.5000+0.5000$i$ & -0.3125+0.7500$i$ & -1.2744+0.5000$i$ \\
$z_2$ & -0.5000+1.0000$i$ & -0.9648+0.0313$i$ & 0.8741-0.7744$i$ \\
$z_3$ & -1.2500-0.5000$i$ & 0.4210+0.4397$i$ & -0.3357-0.8538$i$ \\
$z_4$ & 0.8125+1.7500$i$ & -0.5085+0.8781$i$ & -1.1163+1.0732$i$ \\
$z_5$ & -2.9023+3.3438$i$ & -1.0125-0.3930$i$ & -0.4056-1.8960$i$ \\
$z_6$ & -3.2571-18.9094$i$ & 0.3707+1.2958$i$ & -3.9302+2.0377$i$ \\
$z_7$ & -347.4578+123.6784$i$ & -2.0416+1.4607$i$ & 10.7942-15.5173$i$ \\
$z_8$ &  & 1.5346-5.4646$i$ & -124.7697-334.4940$i$ \\
$z_9$ &  & -28.0066-16.2717$i$ &  \\
\bottomrule
\end{tabular}
