\UseRawInputEncoding
\begin{tabular}{lrr}
\toprule
  c[k] &  successi�\_c[k] &  m�dul\_c[k] \\
\midrule
  c[0] & -0.5000+0.5000$i$ &      0.7071 \\
  c[1] & -0.5000+0.0000$i$ &      0.5000 \\
  c[2] & -0.2500+0.5000$i$ &      0.5590 \\
  c[3] & -0.6875+0.2500$i$ &      0.7315 \\
  c[4] & -0.0898+0.1562$i$ &      0.1802 \\
  c[5] & -0.5163+0.4719$i$ &      0.6995 \\
  c[6] & -0.4561+0.0127$i$ &      0.4563 \\
  c[7] & -0.2921+0.4885$i$ &      0.5692 \\
  c[8] & -0.6533+0.2146$i$ &      0.6876 \\
  c[9] & -0.1193+0.2196$i$ &      0.2499 \\
 c[10] & -0.5340+0.4476$i$ &      0.6968 \\
 c[11] & -0.4152+0.0220$i$ &      0.4158 \\
 c[12] & -0.3281+0.4817$i$ &      0.5829 \\
 c[13] & -0.6244+0.1839$i$ &      0.6509 \\
 c[14] & -0.1439+0.2704$i$ &      0.3063 \\
 c[15] & -0.5524+0.4222$i$ &      0.6953 \\
 c[16] & -0.3731+0.0336$i$ &      0.3746 \\
 c[17] & -0.3619+0.4749$i$ &      0.5971 \\
 c[18] & -0.5946+0.1562$i$ &      0.6148 \\
 c[19] & -0.1709+0.3142$i$ &      0.3577 \\
\bottomrule
\end{tabular}
%\caption{Taula amb la successi� del nombre complex -0.5+0.5$i$ i els seus respectius m�duls. Es pot veure que en 19 iteracions tots aquests m�dul s�n menors que dos.}
