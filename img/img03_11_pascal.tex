\documentclass[border=0.25in]{standalone}

\usepackage{tikz,fp}

\newlength{\R}
\setlength{\R}{.75cm}
\newcommand\mycolor{red!100!black}

%% https://tex.stackexchange.com/questions/34424/how-do-i-calculate-n-modulo-3-in-latex/34425#34425
\newcommand{\modulo}[2]{%
  \FPeval{\result}{trunc(#1-(#2*trunc(#1/#2,0)),0)}\result%
}
\begin{document}
%%  #1 determines which hexagons are filled (default is 2); #2 is number of rows
\newcommand{\makesierp}[2][5]{%
    \begin{tikzpicture}[line width=.8pt]
        \foreach \k in {0,...,#2}{
            \begin{scope}[shift={(-60:{sqrt(3)*\R*\k})}]
                \pgfmathtruncatemacro\ystart{#2-\k}
                \foreach \n in {0,...,\ystart}{
                    \pgfmathtruncatemacro\newn{\n+\k}
                    \pgfmathsetmacro{\myvalue}{1}
                    \ifnum\k>0
                        \foreach \b in {1,...,\k}{
                            \FPeval\myvalue{trunc(\myvalue*(\newn-\b+1)/\b:0)}
                            \global\let\myvalue=\myvalue
                        }
                    \fi
                    \modulo{\myvalue}{#1}%
                    \ifnum\result=0 \def\mycolor{white!60!black}\fi%
                    \begin{scope}[shift={(-120:{sqrt(3)*\R*\n})}]
                        \draw[fill=\mycolor!30] 
                        (30:\R) \foreach \x in {90,150,...,330} {
                        -- (\x:\R)}
                        --cycle (90:0);%node[font=\tiny] {$\mathbf{\myvalue}$};
                    \end{scope}
                }
            \end{scope}
        }
    \end{tikzpicture}
}
\newcommand{\makesierp}[2][3]{%
    \begin{tikzpicture}[line width=.8pt]
        \foreach \k in {0,...,#2}{
            \begin{scope}[shift={(-60:{sqrt(3)*\R*\k})}]
                \pgfmathtruncatemacro\ystart{#2-\k}
                \foreach \n in {0,...,\ystart}{
                    \pgfmathtruncatemacro\newn{\n+\k}
                    \pgfmathsetmacro{\myvalue}{1}
                    \ifnum\k>0
                        \foreach \b in {1,...,\k}{
                            \FPeval\myvalue{trunc(\myvalue*(\newn-\b+1)/\b:0)}
                            \global\let\myvalue=\myvalue
                        }
                    \fi
                    \modulo{\myvalue}{#1}%
                    \ifnum\result=0 \def\mycolor{white!60!black}\fi%
                    \begin{scope}[shift={(-120:{sqrt(3)*\R*\n})}]
                        \draw[fill=\mycolor!30] 
                        (30:\R) \foreach \x in {90,150,...,330} {
                        -- (\x:\R)}
                        --cycle (90:0);%node[font=\tiny] {$\mathbf{\myvalue}$};
                    \end{scope}
                }
            \end{scope}
        }
    \end{tikzpicture}
}
\makesierp{10}
\end{document}
