\RequirePackage{luatex85}
\documentclass[border=5pt]{standalone}
\usepackage{tikz,pgfplots}
\pgfplotsset{compat=1.11}
\usepackage{luamplib}
\begin{document}
\begin{tabular}{cc}
\begin{mplibcode}
def gasket_sierpinski(expr x, y, d, n) = 
    if n > 0:
        gasket_sierpinski(x, y, d/2, n-1);
        gasket_sierpinski(x+d/2, y, d/2, n-1);
        gasket_sierpinski(x, y+d/2, d/2, n-1);
    else: fill (x, y) -- (x+d, y) -- (x, y+d) -- cycle; fi
enddef;
beginfig(2);
    numeric u,v;
    u = 0.5cm;
    v=8;
	%for i = 0 upto 2:
         draw image(gasket_sierpinski(0, 0, 4cm, 5)) ;
    %endfor;
    for i:=0 upto v:
    	draw (0,i*u)--(v*u,i*u);
    	draw (i*u,0)--(i*u,v*u);
    endfor;
endfig;
\end{mplibcode}
&
\begin{mplibcode}
def gasket_sierpinski(expr x, y, d, n) = 
    if n > 0:
        gasket_sierpinski(x, y, d/2, n-1);
        gasket_sierpinski(x+d/2, y, d/2, n-1);
        gasket_sierpinski(x, y+d/2, d/2, n-1);
    else: fill (x, y) -- (x+d, y) -- (x, y+d) -- cycle; fi
enddef;
beginfig(1);
    numeric u;
    u = 0.25cm;
	v=16;
	%for i = 0 upto 2:
         draw image(gasket_sierpinski(0, 0, 4cm, 5)) ;
    %endfor;
    for i:=0 upto v:
    	draw (0,i*u)--(v*u,i*u);
    	draw (i*u,0)--(i*u,v*u);
    endfor;
endfig;
\end{mplibcode}
\\
$r_1=\frac{1}{8}$, $N(r_1)=27$ & $r_2=\frac{1}{16}$, $N(r_2)=81$
\\
&
\\
\begin{mplibcode}
def gasket_sierpinski(expr x, y, d, n) = 
    if n > 0:
        gasket_sierpinski(x, y, d/2, n-1);
        gasket_sierpinski(x+d/2, y, d/2, n-1);
        gasket_sierpinski(x, y+d/2, d/2, n-1);
    else: fill (x, y) -- (x+d, y) -- (x, y+d) -- cycle; fi
enddef;
beginfig(2);
    numeric u;
    u = 0.125cm;
	v=32;
	%for i = 0 upto 2:
         draw image(gasket_sierpinski(0, 0, 4cm, 5)) ;
    %endfor;
    for i:=0 upto v:
    	draw (0,i*u)--(v*u,i*u);
    	draw (i*u,0)--(i*u,v*u);
    endfor;    
endfig;
\end{mplibcode}
&
\begin{mplibcode}
def gasket_sierpinski(expr x, y, d, n) = 
    if n > 0:
        gasket_sierpinski(x, y, d/2, n-1);
        gasket_sierpinski(x+d/2, y, d/2, n-1);
        gasket_sierpinski(x, y+d/2, d/2, n-1);
    else: fill (x, y) -- (x+d, y) -- (x, y+d) -- cycle; fi
enddef;
beginfig(1);
    numeric u;
    u = 0.0625cm;
	v=64;
	%for i = 0 upto 2:
         draw image(gasket_sierpinski(0, 0, 4cm, 5)) ;
    %endfor;
    for i:=0 upto v:
    	draw (0,i*u)--(v*u,i*u);
    	draw (i*u,0)--(i*u,v*u);
    endfor;
endfig;
\end{mplibcode}
\\
$r_3=\frac{1}{32}$, $N(r_3)=243$ & $r_4=\frac{1}{64}$, $N(r_4)=729$
\\
\\
\multicolumn{2}{c}{
\begin{tikzpicture}
    \begin{axis}[legend pos=north west,
    samples=50, 
    domain=2:4.5, 
    ymin=3,
    ymax=7,
    xmin=2, 
    xmax=4.5, 
    xlabel=$\log(1/r)$,
    ylabel=$\log N(r)$, 
    grid=major]
      \addplot [red,smooth, thick] { 1.584963*x };  
      \addplot[blue!50!green,only marks] coordinates {
       (2.07944154, 3.29583687)
       (2.77258872, 4.39444915)
       (3.4657359 , 5.49306144)
       (4.15888308, 6.59167373)
        }; 
      
       \legend{$f(x)=1.584963x$}   
       
     \end{axis}
\end{tikzpicture}
}
\end{tabular}
\end{document}