\RequirePackage{luatex85}
\documentclass[border=0.5mm]{standalone}
\usepackage{luamplib}
\begin{document}
\begin{mplibcode}
vardef sirp(expr A,B,C,n)=   %Procédure pour sierpinski
     if n>0:
      sirp( A,1/2[A,B],1/2[A,C], n-1);
      sirp( 1/2[A,B],  B ,1/2[B,C] , n-1);
      sirp( 1/2[A,C],  1/2[B,C],C  , n-1);
    else:
           fill A--B--C--cycle withcolor black;
     fi;

enddef;
beginfig(1)
u:=2cm;
i=4;
z0=(0,0)*u; 
z2=(0.5,(sqrt(3))/2)*u; 
z1=(1,0)*u;
    draw image(sirp(z0,z1,z2,i)); 
endfig;
\end{mplibcode}
\end{document}



%\RequirePackage{luatex85}
%\documentclass[border=2mm]{standalone}
%\usepackage{luamplib}
%\begin{document}
%\begin{mplibcode}
%  vardef Sierpinski_var(expr A, B, C, n) = % The recursive macro 
%    if n = 0: filldraw A--B--C--cycle;
%    else: 
%      save AC, BC, AB; pair AC[], BC[], AB[];
%      AC1 = 1/2[A,C]; 
%      BC1 = 1/2[B,C]; 
%      AB1 = 1/2[A,B];  
%      Sierpinski_var(A, AB1, AC1, n-1); 
%      Sierpinski_var(AB1, B, BC1, n-1); 
%      Sierpinski_var(AC1,BC1,C, n-1);
%    fi
%  enddef;
%  beginfig(1);
%    u = 5cm; n = 4; 
%    pair A, B, C; A = origin; B = u*right; C = u*dir 60;
%    draw image(Sierpinski_var(origin, u*right, u*dir 60, n));
%  endfig;
%\end{mplibcode}
%\end{document}