\RequirePackage{luatex85}
\documentclass[border=2mm]{standalone}
\usepackage{luamplib}
\begin{document}
\begin{tabular}{c}
\begin{mplibcode}
vardef sirp(expr A,B,C,n)=   %Procédure pour sierpinski
     if n>0:
      sirp( A,1/2[A,B],1/2[A,C], n-1);
      sirp( 1/2[A,B],  B ,1/2[B,C] , n-1);
      sirp( 1/2[A,C],  1/2[B,C],C  , n-1);
    else:
           fill A--B--C--cycle withcolor black;
     fi;
enddef;
beginfig(1)
u:=3cm;
pair v; v = 1.1u*right;
z0=(0,0); 
z2=(0.5,(sqrt(3))/2)*u; 
z1=z0 rotatedaround(1[z0,z2], 30);
z3=z1 rotatedaround(1[z1,z2], 30);
for i = 0 upto 1:
    draw image(sirp(z0,z1,z2,i)) shifted (i*v);
    draw image(sirp(z1,z3,z2,i)) shifted (i*v);
    label.bot(textext("\Large $n=" & decimal i & "$"), (1.5cm+3.25cm*i,-5mm));
endfor;
endfig;
\end{mplibcode}
\\
\\
\begin{mplibcode}
vardef sirp(expr A,B,C,n)=   %Procédure pour sierpinski
     if n>0:
      sirp( A,1/2[A,B],1/2[A,C], n-1);
      sirp( 1/2[A,B],  B ,1/2[B,C] , n-1);
      sirp( 1/2[A,C],  1/2[B,C],C  , n-1);
    else:
           fill A--B--C--cycle withcolor black;
     fi;

enddef;
beginfig(2)
u:=3cm;
pair v; v = 1.1u*right;
z0=(0,0); 
z2=(0.5,(sqrt(3))/2)*u; 
z1=z0 rotatedaround(1[z0,z2], 30);
z3=z1 rotatedaround(1[z1,z2], 30);
for i = 2 upto 3:
    draw image(sirp(z0,z1,z2,i)) shifted (i*v);
    draw image(sirp(z1,z3,z2,i)) shifted (i*v);
    label.bot(textext("\Large $n=" & decimal i & "$"), (1.15cm+3.45cm*i,-5mm));
endfor;
%label(TEX("$n=0$"), (1.5cm,-5mm));
%label(TEX("$n=1$"), (4.75cm,-5mm));
endfig;
\end{mplibcode}
\end{tabular}
\end{document}